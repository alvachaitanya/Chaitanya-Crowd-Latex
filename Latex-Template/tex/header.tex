\usepackage{multirow,enumitem,cite,epsfig,epstopdf}
\usepackage{amssymb,graphics,graphicx,algorithm,xcolor}%,algorithmic}
\usepackage{times,color,booktabs,changepage}%,subcaption}
\usepackage[export]{adjustbox}
\usepackage[cmex10]{amsmath}
\usepackage[english]{babel}
\usepackage[colorlinks=true,allcolors=black,urlcolor=blue]{hyperref}

\usepackage[left=3cm, right=3cm]{geometry}
\usepackage{listings} % for source code
\lstset{basicstyle=\small\ttfamily,showstringspaces=false,keywordstyle=\bfseries\color{blue!70!black}}

\usepackage[T1]{fontenc}
\usepackage[default]{opensans} 


\def\Fig#1{Fig.~\ref{fig:#1}}
\def\Eq#1{Eq.~(\ref{eq:#1})}
\def\Tab#1{Table~\ref{tab:#1}}
\def\Sec#1{Section~\ref{sec:#1}}
\def\Prop#1{Proposition~\ref{prop:#1}}

% A comment in a draft (shouldn't appear in the final version).
\newcommand{\comment}[2]{\( \clubsuit\){\bf #1: }{\rm \sf #2}\(\clubsuit\)}
% Comment this out in the draft
%\newcommand{\comment}[2]{}
\newcommand{\dbcomment}[1]{\comment{DB}{#1}}
\newcommand{\rscomment}[2]{\comment{RS}{#1}}
\newcommand{\pgcomment}[2]{\comment{PG}{#1}}
%---------------------------------

\def\eg{\emph{e.g.}, }
\def\Eg{\emph{E.g.}, }
\def\etal{\emph{et al.}}
\def\ie{\emph{i.e.}, }
\def\noi{\noindent}

%-----------------------
% Change the way IEEE write table caption
\usepackage{etoolbox}
\makeatletter
\patchcmd{\@makecaption}
  {\scshape}
  {}
  {}
  {}
\makeatletter
\patchcmd{\@makecaption}
  {\\}
  {.\ }
  {}
  {}
\makeatother
%\def\tablename{Table}
%-------------------------

%%%%%%%%%%%%%%%%%%%%%%%%%%%%%%%%%%%%%%%%%%%%%%%%
% with 12 point font, single line spacing, single-sided pages
\setlength{\parskip}{0.5\baselineskip}
\setlength{\textheight}{23cm}
\setlength{\footskip}{1.5cm}
\usepackage[singlespacing]{setspace} % don't use onehalfspacing
%\usepackage{titlesec}
%\titlespacing*{\chapter}{0pt}{-50pt}{20pt}
%\titleformat{\chapter}[hang]{\normalfont\huge\bfseries}{\thechapter}{1em}{} 

%\newcommand{\unnumberedchapter}[1]{\chapter*{#1}\addcontentsline{toc}{chapter}{#1}}

%-----------------------------------------------------------
% Chapter style definition
%-----------------------------------------------------------
\usepackage[explicit]{titlesec}
\usepackage{lmodern}

\newlength\chapnumb
\setlength\chapnumb{4cm}

\titleformat{\chapter}[block]
{\normalfont\sffamily}{}{0pt}
{\parbox[b]{\chapnumb}{%
		\fontsize{120}{110}\selectfont\thechapter}%
	\parbox[b]{\dimexpr\textwidth-\chapnumb\relax}{%
		\raggedleft%
		\hfill{\bfseries{\LARGE#1}}\\
		\rule{\dimexpr\textwidth-\chapnumb\relax}{0.4pt}}}
	
\titleformat{name=\chapter,numberless}[block]
{\normalfont\sffamily}{}{0pt}
{\parbox[b]{\textwidth}{%
		\bfseries{\huge#1}\\
		\rule{\dimexpr\textwidth\relax}{0.4pt}}
 \parbox[b]{\chapnumb}{%
 	\mbox{}}%
}%

\titlespacing{name=\chapter,numberless}{0em}{0em}{-1.0em} % 
%{left}{before}{after}[right]

\newcommand{\cmd}[1]{\texttt{\textbackslash{}#1}}
\newcommand{\unnumberedchapter}[1]{\chapter*{#1}\addcontentsline{toc}{chapter}{#1}}
%-----------------------------------------------------------



\let\tempit\itemize \let\tempeit\enditemize 
\renewenvironment{itemize}{\vspace{-1em}\tempit\addtolength{\itemsep}{-0.5\baselineskip}}{\tempeit}
% Changes itemize spacing
\let\tempen\enumerate \let\tempeen\endenumerate 
\renewenvironment{enumerate}{\vspace{-1em}\tempen\addtolength{\itemsep}{-0.5\baselineskip}}{\tempeen}
% Changes enumerate spacing
%%%%%%%%%%%%%%%%%%%%%%%%%%%%%%%%%%%%%%%%%%%%%%%%
