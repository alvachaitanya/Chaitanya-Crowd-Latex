\chapter{Results and Discussions}
\section{Dataset}
Description of the dataset(s)

\section{Experimental setup}
Say what is the experimental set up, parameters that were used. 

\section{Results}
Stand back and evaluate what you have achieved and how well you have met the 
objectives. Evaluate your achievements against your objectives in 
\Sec{objectives}. Demonstrate that you have tackled the project in a 
professional manner. 

The previous paragraph demonstrates the use of automatic cross-references:  The 
``1.2'' is a \textit{cross-reference} to the text in a numbered item of the 
document; you do not type it as \texttt{1.2} but by using the \cmd{Sec} 
command. The number that appears here will change automatically if the number 
on the referred-to section is altered, for example, if a chapter or section is 
added or deleted before it. Cross-references to section are entered with the 
\cmd{ref} command just like for figures. The \TeX{} code above reads
\begin{quote}\tt
	Evaluate your achievements against your objectives \\[-0.5em]
	in section \textbackslash{}ref\{objectives sec\}.
\end{quote}
For this to work, the code for the text on page \pageref{objectives sec} must 
read
\begin{quote}\tt
	\textbackslash{}section\{Scope and Objectives\} 
	\textbackslash{}label\{objectives sec\}
\end{quote}
As with figure labels, the text inside of \cmd{label} and \cmd{Fig} never 
appears in the final pdf; you can make it whatever you want as long as you use 
the same text in each to complete the reference.

\section{Discussions}
Analyse your results and discuss it by including your insight. For example why 
the results are behaving like this, why there is an outlier etc.