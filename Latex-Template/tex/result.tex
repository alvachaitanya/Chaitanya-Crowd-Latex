\chapter{Results and Discussions}

\section{Environment}
The whole project is implemented on 2 environments. One is to build the datasets from the video footages. Second is to train and test the CNN and ML models. 
The environment setup for the first scenario is displayed in the table \ref{tab:scenario1}. The environment setup for the second scenario is displayed in the table  \ref{tab:scenario2}.  
\begin{table}[h!]
\centering
 \begin{tabular}{| p{3cm} p{3cm} p{6cm}|} 
 \hline
 \textbf{Software/ Libraries} & \textbf{Version}  & \textbf{Reason} \\
 \hline\hline
 PyCharm & 2020.2 & IDE to develop software \\
 \hline
 Python & 2.7.16 & Programming language \\
 \hline
 opencv-python & 4.2.0.34 & Optical Flow \\
 \hline
 NumPy & 1.18.4 & Matrix operations \\
 \hline
 Scikit-learn & 0.23.1 & Density based clustering \\
 \hline
 Pandas & 1.0.5 & Dataframe and CSV files \\
 \hline
 Matplotlib & 3.2.2 & Plotting images  \\
 \hline
 Pillow & 7.1.2 & Converting plots to images \\
 \hline
\end{tabular}
\caption{Requirements to model Implementation}
\label{tab:scenario1}
\end{table}

\begin{table}[h!]
\centering
 \begin{tabular}{| p{3cm} p{3cm} p{6cm}|} 
 \hline
 \textbf{Software/ Libraries} & \textbf{Version}  & \textbf{Reason} \\
 \hline\hline
 Google Colab &  & Notebook \\
 \hline
 Python & 3.6 & Programming language \\
 \hline
 torch & 1.6.0 & CNN models \\
 \hline
 NumPy & 1.18.4 & Matrix operations \\
 \hline
 Scikit-learn & 0.23.1 & Machine Learning Models\\
 \hline
 Pandas & 1.0.5 & CSV files \\
 \hline
 Matplotlib & 3.2.2 & Plotting images  \\
 \hline
\end{tabular}
\caption{Requirements for CNN and ML models Implementation}
\label{tab:scenario2}
\end{table}

\section{Experimental setup}
There are 3 sets of experiments conducted while implementing and testing the model. Different tests in each set are conduction in the below format :

\subsection{Experiments on Optical Flow }
 \begin{itemize}
	\item Test the quality and performance of Feature detection techniques.
	\item Test the quality and performance of Density based clustering.
	\item Test the performance by changing the Block Size.
	\item Test the quality of images with or with out magnitude multiplication.
\end{itemize}
\subsection{Experiments on CNN models }
 \begin{itemize}
	\item Test different hyper-parameters to increase the accuracy.
	\item Test the models with change in epochs.
\end{itemize}
\subsection{Experiments on ML models }
 \begin{itemize}
	\item Test different hyper-parameters to increase the accuracy.
\end{itemize}

\section{Results}
Stand back and evaluate what you have achieved and how well you have met the 
objectives. Evaluate your achievements against your objectives in 
\Sec{objectives}. Demonstrate that you have tackled the project in a 
professional manner. 

The previous paragraph demonstrates the use of automatic cross-references:  The 
``1.2'' is a \textit{cross-reference} to the text in a numbered item of the 
document; you do not type it as \texttt{1.2} but by using the \cmd{Sec} 
command. The number that appears here will change automatically if the number 
on the referred-to section is altered, for example, if a chapter or section is 
added or deleted before it. Cross-references to section are entered with the 
\cmd{ref} command just like for figures. The \TeX{} code above reads
\begin{quote}\tt
	Evaluate your achievements against your objectives \\[-0.5em]
	in section \textbackslash{}ref\{objectives sec\}.
\end{quote}
For this to work, the code for the text on page \pageref{objectives sec} must 
read
\begin{quote}\tt
	\textbackslash{}section\{Scope and Objectives\} 
	\textbackslash{}label\{objectives sec\}
\end{quote}
As with figure labels, the text inside of \cmd{label} and \cmd{Fig} never 
appears in the final pdf; you can make it whatever you want as long as you use 
the same text in each to complete the reference.

\section{Discussions}
Analyse your results and discuss it by including your insight. For example why 
the results are behaving like this, why there is an outlier etc.