\chapter{Feature Detection}
Feature detection using open CV is the a very useful and important technique is most of the areas which deals with images and videos. Each image or the frame is the combination of pixels and each each pixel is the number that represents a colour. For a computer it is extremely difficult to understand the difference between these numbers and thus, feature detection is a complex yet interesting topic to understand. This paper tries to implements 3 different corner detection techniques 1) Harris Corner detection, 2) Shi Tomasi corner detection. 3) FAST algorithm for corner detection. The advantages and disadvantages of these techniques are discussed further.
\section{Harris Corner Detection}
 This corner detection technique was first introduced by Chris Harris \& Mike Stephens in their paper ~\cite{harris1988combined} in 1988. Idea behind this technique is to find the difference between the intensity for a displacement of (u, v) in all directions. The mathematical equation \ref{eqn:1} for the same is given below.
 
 \begin{equation}\label{eqn:1}
    E(u, v) = \sum_{x, y} w(x, y) [I(x+u, y+v)-I(x, y)]^{2}
 \end{equation}

Window function \textit{w(x, y)} is either a rectangular window or gaussian window which gives weights to pixels underneath. The corners are detected by maximising the \textit{E(u, v)} which means maximising the second term by using Tylor's theorem as shown in the equations \ref{eqn:2} \& \ref{eqn:3}

 \begin{equation}\label{eqn:2}
    	E(u, v) \approx 
	\begin{bmatrix} 
		u & v
	\end{bmatrix} 
	M
	\begin{bmatrix}
		u\\v 
	\end{bmatrix}
\end{equation}

where

\begin{equation}\label{eqn:3}
	M = \sum_{x,y} w(x, y)
    	\begin{bmatrix}
		I_{x}I_{x} & I_{x}I_{y}
		\\
		I_{x}I_{y} & I_{y}I_{y}
	\end{bmatrix}
\end{equation}
Here, $I_{x}$ and $I_{y}$ are image derivatives in x and y directions respectively.
\textit{R} value is calculated from the eigenvalues of the matrix \textit{M} from the equation \ref{eqn:4}
 \begin{equation}\label{eqn:4}
	R = det(M) - K (trace(M))^{2}
\end{equation}
where
\begin{itemize}
	\item det(M) = $\lambda_{1} \lambda_{2}$.
	\item trace(M) = $\lambda_{1} + \lambda_{2}$.
	\item $\lambda_{1}$ and $\lambda_{2}$ are the eigenvalues of M.
	\item k - Harris detector free parameter in the equation.
\end{itemize}
edge, corner and flat region in the image are detected with the help of eigenvalues as shown in the figure \ref{fig:harris}

\begin{figure}[tb]
	\center\includegraphics[width=0.6\textwidth]{harris.jpg}
	\caption{Harris corner detection using eigenvalues}
	\label{fig:harris}
\end{figure}
\section{Shi Tomasi Corner Detection}
Shi Tomasi Corner detection was first proposed by J. Shi and C. Tomasi in the paper ~\cite{shi1994good} in 1994. This approach is a small modification to the Harris Corner detection in calculating the \textit{R} value. As mentioned before the \textit{R} value is calculated by \ref{eqn:4}
But, as per Shi Tomasi Corner detection, the \textit{R} value is calculated by minimising the product of eigenvalues as shown in \ref{eqn:5}.
 \begin{equation}\label{eqn:5}
	R = min(\lambda_{1}, \lambda_{2})
\end{equation}
If this value is greater than the threshold, then it is considered as the corner.
\section{FAST Algorithm for Corner Detection}




\chapter{Density Based Clustering}

\chapter{Lucas-Kanade Algorithm}

\chapter{Convolutional Neural Networks}

\chapter{Machine Learning}

\chapter{Data collection \& Pre Processing}

\chapter{Implementation}
The technical body of the dissertation consists of a number of chapters (just 
one here, but there will usually be more).  Follow a logical structure in how 
you present your work.  This will usually be the phases of the software 
development cycle, the modules of your system, etc. \textbf{\textit{However, 
please do not write your dissertation to read like a diary.}}

Include a chapter demonstrating what you have achieved and how your system is 
used in practice -- for example showing a typical session as a series of pasted 
in screen shots, with an accompanying commentary.

You \textbf{\textit{should}} also include a chapter explaining how you obtained 
feedback from your ``customer'' or potential users of your system, what 
feedback you actually obtained, and your analysis and comments.

\section{First Section}
Subdivide your text into sections, using the \cmd{section} command.

\subsection{First Subsection}
If necessary, also use subsections. Subsections are entered using the 
\cmd{subsection} command (all these heading styles are self-numbering).

%\subsubsection{First Subsubsection}
%If you really need subsubsections, enter these using the \cmd{subsubsection} 
%command. Note that subsubsections in \TeX{} do not display numbers or appear 
%in 
%the Table of Contents (they just show a bold header on its own line).

\subsection{Second Subsection}
And, as required, more subsections.

\section{Bulleted and Numbered Lists}
Note: This section begins with the code 
\texttt{\textbackslash{}section\{Bulleted and Numbered Lists\}} in 
the~\texttt{.tex} file.

Bulleted or numbered lists are entered using the \texttt{itemize} and 
\texttt{enumerate} environments, respectively. An \textbf{environment} in 
\LaTeX{} is a block of code in between a \cmd{begin} and \cmd{end} command. For 
example, the code
\begin{quote}\tt
	\textbackslash{}begin\{itemize\} \\[-0.5em]
	\hspace*{2em}\textbackslash{}item Up \\[-0.5em]
	\hspace*{2em}\textbackslash{}item Down \\[-0.5em]
	\hspace*{2em}\textbackslash{}item Left \\[-0.5em]
	\hspace*{2em}\textbackslash{}item Right \\[-0.5em]
	\textbackslash{}end\{itemize\}
\end{quote}
would produce the following list:
\begin{itemize} \item Up \item Down \item Left \item Right \end{itemize}
The indentation is not necessary (the pdf will look the same even it the 
\texttt{.tex} file does not use indents), but it helps make the code easier to 
read.

If the \texttt{enumerate} environment is used instead, the bullets are replaced 
by numbers. For example, the code
\begin{quote}\tt
	\textbackslash{}begin\{enumerate\} \\[-0.5em]
	\hspace*{2em}\textbackslash{}item Up \\[-0.5em]
	\hspace*{2em}\textbackslash{}item Down \\[-0.5em]
	\hspace*{2em}\textbackslash{}item Left \\[-0.5em]
	\hspace*{2em}\textbackslash{}item Right \\[-0.5em]
	\textbackslash{}end\{enumerate\}
\end{quote}
produces the list
\begin{enumerate} \item Up \item Down \item Left \item Right \end{enumerate}

\section{Figures and Captions}
As an example of a figure, consider Figure \ref{fig:mylovelydiagram}. Captions are 
entered using the \texttt{figure} environment (read the previous section for 
information about environments in general). The code
\begin{quote}\tt
	\textbackslash{}begin\{figure\}[h] \\[-0.5em]
	\hspace*{2em}\textbackslash{}center\textbackslash{}includegraphics[width=12cm]\{image.png\}
	 \\[-0.5em]
	\hspace*{2em}\textbackslash{}caption\{Highly Technical Diagram\} \\[-0.5em]
	\hspace*{2em}\textbackslash{}label\{mylovelydiagram\} \\[-0.5em]
	\textbackslash{}end\{figure\}
\end{quote}
will produce the following figure \textbf{if the file \textit{image.png} is in 
the same folder as your \texttt{.tex} file}.

\begin{figure}[tb]
	\center\includegraphics[width=0.6\textwidth]{image.jpg}
	\caption{Highly Technical Diagram}
	\label{fig:mylovelydiagram}
\end{figure}


\begin{figure}[tb]
	\center\includegraphics[width=0.6\textwidth]{image.jpg}
	\caption{Highly Technical Diagram two}
	\label{fig:mylovelydiagramtwo}
\end{figure}

The \texttt{[tb]} direction after the beginning of the environment causes the 
figure to be placed ``here'' in the text (at least approximately -- sometimes 
\TeX{} will move the figure slightly if the spacing does not work well in 
exactly the given location). For large figures, use \texttt{[t]} 
or~\texttt{[b]} instead to place the figure at the top or bottom of a page. You 
can also leave off the \texttt{[h]} entirely to have \TeX{} make its best guess 
for where the figure should go.

The \cmd{includegraphics} command puts an image file from your computer into 
your finished pdf. \textbf{If there is no file with the given name in the 
folder with your \texttt{.tex} file, your document will not compile at all.} 
The bracket text \texttt{[width=12cm]} is optional; without it, \TeX{} will use 
the normal size of the image. Sometimes this will be far too large, so it is a 
good idea to specify a width directly.

Figures have automatic numbering, and it is possible to make cross-references 
to figures. The code \cmd{Fig\{mylovelydiagram\}} will create a link to 
\Fig{mylovelydiagram} in the text with the number of that figure. You can 
change the text ``mylovelydiagram'' to be anything you want -- it never appears 
in the final pdf.

\section{Source Code}

To include programming source code in your document, use the 
\texttt{lstlisting} environment. The \LaTeX{} code
\begin{quote}\tt
	\textbackslash{}begin\{lstlisting\}[language=Python, frame=single] 
	\\[-0.5em]
    \hspace*{2em}def factorial(n): \\[-0.5em]
    \hspace*{4em}if n == 0: return 1 \\[-0.5em]
    \hspace*{4em}else: return n * factorial(n-1) \\[-0.5em]
    \textbackslash{}end\{lstlisting\}
\end{quote}
produces the following in the pdf: \\
\begin{lstlisting}[language=Python, frame=single, label={lst:label}, 
caption={Some Python code}]
def factorial(n):
	if n == 0: return 1
	else: return n * factorial(n-1)
\end{lstlisting}
You can change \texttt{language=Python} to \texttt{language=Java}, etc., for 
different programming languages. The \texttt{frame=single} can be removed if 
you do not want the border around your code snippet. See 
\url{https://en.wikibooks.org/wiki/LaTeX/Source_Code_Listings} for syntax 
coloring and other option. You can reference the listing with the command, 
\texttt{\textbackslash{}ref\{lst:label\}}, as in see listing \ref{lst:label}.
