\chapter{Introduction}
\section{General Context and motivation}
Evolution of technology, artificial intelligence and robotics helped the world to achieve new targets in the
field of security and surveillance. The combination of machine learning and surveillance emerged as a
powerful tool to tackle crime, illegal activities and violent protests. In the recent years, we experienced 
many such activities that made us to understand the importance and necessity of automated video 
surveillance. With the help of computer vision, detecting people in the frame, counting the people in a 
dense scene, abnormal behaviour detection and motion analysis in surveillance videos is done without 
any manual intervention. Crowd motion analysis and abnormal behaviour detection have always been a 
challenging task in this field. Reason being the number of independent factors that define the motion of 
the individual. Analysing the motion of the crowd can avoid many voluntary or involuntary violence, riots, 
traffic jams and stampede.

As mentioned in~\cite{sreenu2019intelligent}, the main objectives of automated surveillance video 
analysis are continuous monitoring, reduction in laborious human task, object identification or action 
recognition and crowd analysis. This paper talks about detecting different types of crowd motion and abnormal behaviour tracking using CNN. Understanding the following characteristics of the video helps to choose the best approach for the motion analysis.
\begin{itemize}
	\item Crowd type based on density.
	\item Detecting the motion in the frame.
	\item Identifying the types of motion.
	\item Categorising into normal or abnormal behaviour.
\end{itemize}
\subsection{Crowd Type based on density}
It is important to categorise the type of the crowd to understand the dynamics of the motion. As described by Moore, Brian~\cite{moore2011visual}, the crowd can be visualised in hydrodynamic point of view. They view the crowd to be in 3 types, microscopic, mesoscopic and macroscopic based on the density. 

\section{Aim and Objectives} \label{sec:objectives}
Define the scope and objectives of your project.

\section{Achievements}
Summarise what this project has achieved. Avoid terms like I achieved this or 
that. 

\section{Overview of Dissertation}
Briefly overview the contents of what follows in the dissertation.