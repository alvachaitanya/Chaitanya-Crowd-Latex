\chapter{Introduction}
Evolution of technology, artificial intelligence and robotics helped the world to achieve new targets in the
field of security and surveillance. The combination of machine learning and surveillance emerged as a
powerful tool to tackle crime, illegal activities and violent protests. In the recent years, we experienced 
many such activities that made us to understand the importance and necessity of automated video 
surveillance. With the help of computer vision, detecting people in the frame, counting the people in a 
dense scene, abnormal behaviour detection and motion analysis in surveillance videos is done without 
any manual intervention. Crowd motion analysis and abnormal behaviour detection have always been a 
challenging task in this field. Reason being the number of independent factors that define the motion of 
the individual. Analysing the motion of the crowd can avoid many voluntary or involuntary violence, riots, 
traffic jams and stampede.
\section{General Context and motivation}
As mentioned in~\cite{sreenu2019intelligent}, the main objectives of automated surveillance video 
analysis are continuous monitoring, reduction in laborious human task, object identification or action 
recognition and crowd analysis. This paper talks about detecting different types of crowd motion and 
abnormal behaviour tracking using CNN. Most of the study on analysing the crowd is done on the 
following areas.
\begin{itemize}
	\item Counting the crowd.
	\item Types of the crowd based on density.
	\item Detecting the motion in the frame.
	\item Identifying the types of motion.
\end{itemize}

\subsection{Counting the crowd.}
Counting the crowd is a very important in order to maintain the safety and security. It helps to plan the 
events, traffic and the capacity of any situation. But, counting dense crowd is a difficult task. As 
mentioned in~\cite{tripathi2019convolutional}, more than 17\% of the total papers written on crowd 
analysis are published on crowd counting. For example, ~\cite{8462345} generalise different types of 
crowd counting and different algorithms used in the past while proposing a new approach of using the 
statistics of thespatio-temporal wavelet sub-bands. ~\cite{idrees2013multi} uses a multi source 
(identifying different parts of the body in the frames from different algorithms) and Markov Random Field 
to count the people in the dense crowd.
\subsection{Types of the crowd based on density}
It is important to categorise the type of the crowd to understand the dynamics of the motion. 
Moore~\cite{moore2011visual} suggests, the crowd can be treated as particles in fluid dynamics and the 
crowd is of 3 types, microscopic, mesoscopic and macroscopic based on the density. Microscopic view 
of crowd through a hydrodynamic lens implies understanding the flow of every individual in crowd and 
this is specific to limited number of individuals in the frame. Mesoscopic view implies more number of 
people in a frame. Macroscopic view implies the frame filled with people. The personal and interaction 
forces in each case are different which in turn drive the motion of the crowd. To further explain, the 
interaction force is very less in a microscopic view but very high in macroscopic view.
\subsection{Detecting the motion in the frame}
Detecting the motion in the frame can be done either by training a model which involves feeding the 
motion images into a CNN architecture or without training by just tracking every point in the frame using 
optical flow. Santoro~\cite{santoro2010crowd} did optical flow computation with the help of Shi-Tomasi 
Corner Detection and Lucas–Kanade algorithm to detect the motion of the crowd. Where as 
~\cite{9078065} uses motion information Images (MII) to train a CNN model for the motion and abnormality detection
\subsection{Identifying the types of motion}
Identifying the types of the motion can be a very useful in order to understand the crowd behaviour, 
planning an event, avoiding traffic jams and predicting the abnormal motion. Wei~\cite{wei2020very} trained 2 VGG16 CNN 
architecture models to detect the type of the crowd whether it is homogenous, heterogeneous or violent crowd. ~\cite{solmaz2012identifying} s
tudies the stability with the help of Tylor's theorem and Jacobean matrix and identifies the crowd motion to be of 5 generic 
types i.e. Lanes, arc/circle, fountainheads, bottlenecks and blocks.

\section{Aim and Objectives} \label{sec:objectives}
This project aims to classify different types of crowd motion into 4 classes: Arcs, Lanes, Converging/Diverging and Blocks/Random. The project focuses on the drone footages by exploring state of the art CNN and machine learning techniques. 
This project also explore different optical flow techniques and compare the advantages and disadvantages of the existing techniques. The proposed model also helps to understand the motion in the scene and can be used to train multiple anomaly detection techniques. To achieve these aims the following objectives were setup:
\begin{enumerate}
	\item Identifying the interesting features in the frame to track the motion with the help of different corner detection techniques.
	\item Exploring multiple options for noise reduction and improving the quality of the tracking through density based clustering algorithms.
	\item Developing a new approach to track the points which improves the quality of the features and reduce the computation time.
	\item Use of this approach to understand the motion in the scene and create the data for anomaly detection.
	\item Creating the Motion information image dataset from multiple videos and labelling the data for training CNN model.
	\item Creating a Spacio Temporal dataset with block wise information of magnitude and direction for training the machine learning models.
	\item Comparing the best approach visually and statistically to identify the best approach in this context.
	\item Improving the model to perform similar operation from a fixed lens security cameras to a dynamic drone footages.
\end{enumerate}

\section{Achievements}
Summarise what this project has achieved. Avoid terms like I achieved this or 
that. 

\section{Overview of Dissertation}
This dissertation project is organised into different chapters starting with Introduction to project and motivation to work on this project. Full details on the aims and objectives are discussed in this chapter.

\textit{Chapter 2} : Information and basic knowledge about the various technical details used in the project are discussed here. This chapter focuses on providing technical background to understand the project.

\textit{Chapter 3} : This chapter contains various state of the art techniques in the field of crowd analysis and is organised on different types of research performed.

 \textit{Chapter 4} : There are different types of data which are used to train models in the project. The information about the datasets, data collection and pre processing is explained in this chapter.
 
  \textit{Chapter 5} : Optical flow implementation provides the details on various feature detection techniques, clustering algorithms and the object tracking implementation. This chapter also provides the information on practical implementation of these techniques.
  
  \textit{Chapter 6} : Implementation presents the model architecture diagram and the implementation of the logic, data segregation, noise reduction, CNN and machine learning details.

  \textit{Chapter 7} : Detailed analysis on the results and the evaluation of models based on hyper parameters is discussed in this chapter.
  
    \textit{Chapter 8} : Conclusions and future work summarises the project results and suggests the areas which can be improved to expand the applications of this project.
  
  
 














