\chapter{Introduction}
\section{General Context and motivation}
Evolution of technology, artificial intelligence and robotics helped the world to achieve new targets in the
field of security and surveillance. The combination of machine learning and surveillance emerged as a
powerful tool to tackle crime, illegal activities and violent protests. In the recent years, we experienced 
many such activities that made us to understand the importance and necessity of automated video 
surveillance. With the help of computer vision, detecting people in the frame, counting the people in a 
dense scene, abnormal behaviour detection and motion analysis in surveillance videos is done without 
any manual intervention. Crowd motion analysis and abnormal behaviour detection have always been a 
challenging task in this field. Reason being the number of independent factors that define the motion of 
the individual. Analysing the motion of the crowd can avoid many voluntary or involuntary violence, riots, 
traffic jams and stampede.

As mentioned in~\cite{sreenu2019intelligent}, the main objectives of automated surveillance video 
analysis are continuous monitoring, reduction in laborious human task, object identification or action 
recognition and crowd analysis. This paper talks about detecting different types of crowd motion and 
abnormal behaviour tracking using CNN. Most of the study on analysing the crowd is done on the following areas.
\begin{itemize}
	\item Counting the crowd.
	\item Type of the crowd based on its density.
	\item Detecting the motion in the frame.
	\item Identifying the types of motion.
	\item Categorising into normal or abnormal behaviour.
\end{itemize}

\subsection{Counting the crowd.}

\subsection{Type of the crowd based on its density}
It is important to categorise the type of the crowd to understand the dynamics of the motion. Moore~\cite{moore2011visual} suggests, the crowd can be treated as particles in fluid dynamics and the crowd is of 3 types, microscopic, mesoscopic and macroscopic based on the density. Microscopic view of crowd through a hydrodynamic lens implies understanding the flow of every individual in crowd and this is specific to limited number of individuals in the frame. Mesoscopic view implies more number of people in a frame. Macroscopic view implies the frame filled with people. The personal and interaction forces in each case are different which in turn drive the motion of the crowd. To further explain, the interaction force is very less in a microscopic view but very high in macroscopic view.
\subsection{Detecting the motion in the frame}
Detecting the motion in the frame can be done either by training a model which involves feeding the motion images into a CNN architecture or without training by just tracking every point in the frame using optical flow. Santoro~\cite{santoro2010crowd} did optical flow computation with the help of Shi-Tomasi Corner Detection and Lucas–Kanade algorithm to detect the motion of the crowd. Where as Wei~\cite{wei2020very} trained 2 deep vgg16 cnn architecture models to detect the motion of the crowd.


\section{Aim and Objectives} \label{sec:objectives}
Define the scope and objectives of your project.

\section{Achievements}
Summarise what this project has achieved. Avoid terms like I achieved this or 
that. 

\section{Overview of Dissertation}
Briefly overview the contents of what follows in the dissertation.